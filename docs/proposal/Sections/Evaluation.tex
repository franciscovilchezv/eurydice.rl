Measuring the performance of how well is our model is generating a musical melody can be a little complicated, since the quality of a musical composition is relative to each person's preference \cite{biles2013lessons}. For our personal measurement, we will qualify the different set of compositions generated by individually listening to them and assign a good or bad label, and request other people to assign those labels according to their criteria. In order to compare our melodies with the results of other works that use the same methods, we will do a musical theoretical analysis of the notes that belong to the key and compare it with the metrics from previous works \cite{deeprl2016music}.

It is important to emphasize that the metrics based on music theory will not necessarily illustrate the performance of our algorithm since the learning process depends on the qualification that the user is given to the algorithm, which is not focused on following the theoretical rules, but satisfy its personal musical taste. Because of that, our personal measurement will provide a more interesting guidance for this type of project although it may not be used to compare it with the performance from previous works.