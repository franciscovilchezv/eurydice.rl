The first approach analyzed for making this project was the one by \citeauthor{deeprl2016music} They proposed the usage of multiple deep Q-networks to store the rewards for each note and also for musical theory. That way, their algorithm will have the freedom to create melodies which are based on its experience, but will also take into consideration the different musical rules that will ``ensure'' that their compositions have a pleasant sound. Because of that, the reward function takes into consideration both, the rewards based on experience and the value according the musical rules. For both of them, and also for the target deep Q-network, \emph{Recurrent Neural Networks (RNN)} were used \cite{deeprl2016music}. The usage of \emph{RNN}s for the musical composition problem has been corroborated by \citeauthor{eck2002blues}, specifically with the usage of \emph{Long Short Term Memory (LSTM)} RRNs. As they mentioned, \emph{LSTM} is able to keep characteristics such as timing and musical structure while training, which was not possible with other type of models. They applied the solution to the \emph{blues} musical genre and obtained pleasant results according to their work \cite{eck2002blues}. Both proposals based their work on optimizing the best note that should follow a previous sequence of notes, which is considered the main goal in the musical composition problem \cite{connectionist1989}. Our project differs from these two principal approaches in three aspects. Firstly, the usage of musical theory is completely ignored, since we consider that, even though musical theory can explain why a melody sounds nice, it should not constraint the set of notes that can be used \cite{vilchez2015genetic} \cite{biles2013lessons}. Secondly, we will try to introduce a novel data structure which is based on the note's grades instead of storing a note itself (more details in the next section). Finally, the principal goal of our project is to enable the human-computer interaction for the training process and that way we expect the computer to develop a more human-like musical skill.